\documentclass[]{elsart}

% Letzte �nderungen von: $Author: suermeli $ am $Date: 2008-02-29 14:15:11 $

\usepackage{url}
\usepackage{amsmath}
\usepackage{bbm}
\usepackage[latin1]{inputenc}

\usepackage{graphicx}
\usepackage{subfigure}
\usepackage[outercaption]{sidecap}
\usepackage{xspace}

\journal{DKE}

%\setlength{\intextsep}{2mm}%{6mm plus 2mm minus 2mm}

\newcommand{\abb}[1]{{\small\textsf{#1}}\xspace}
%\newcommand{\abb}[1]{\textsf{#1}}

\begin{document}

\newlength{\abovecaptionskip}
\setlength{\abovecaptionskip}{0pt}
\newlength{\belowcaptionskip}
\setlength{\belowcaptionskip}{0pt}

\begin{frontmatter}

\title{Fiona - Functional Interaction Analysis for Open Workflow Nets}
\subtitle{User's Manual}

\author{Peter Massuthe},  \ead{massuthe@informatik.hu-berlin.de}%
\author{Daniela Weinberg} \ead{weinberg@informatik.hu-berlin.de}%

\address{Humboldt-Universit\"at zu Berlin, Institut f\"ur Informatik,\\ Unter den Linden 6, 10099 Berlin,
Germany}

\begin{abstract}
This manual is for \textit{Fiona}, Version \textit{VERSION}. \textit{Fiona} is a tool to
automatically analyze the interactional behavior of a given open workflow net (\textit{oWFN}). This
manual does not explain how to setup or install \textit{Fiona}. For this information please read
the Installation Manual which is part of the distribution or can be downloaded from the website of
\textit{Fiona} (\textit{http://www.informatik.hu-berlin.de/top/tools4bpel/fiona}). Last update
\textit{UPDATED}.

Copying and distribution of this file, with or without modification, are permitted in any medium
without royalty provided the copyright notice and this notice are preserved. \textit{Fiona} is
licensed under the \textit{GNU} General Public License.

Copyright \copyright 2005, 2006, 2007 Daniela Weinberg, Peter Massuthe, Karsten Wolf, Kathrin
Kaschner, Christian Gierds and Jan Bretschneider.

\end{abstract}

\begin{keyword}
dsf
\end{keyword}

\end{frontmatter}


\section{Overview}
\subsection{Introduction}%
\textit{Fiona} is a tool to automatically analyze the interactional behavior of a given open
workflow net (\textit{oWFN}) [MRS05]. It provides two techniques:

\begin{itemize}
\item it checks for the controllability of the given net by computing the interactiong
graph [Wei06],
\item it calculates the operating guideline [MRS05] for the net.
\end{itemize}

Fiona uses oWFNs as its input which is the output of the tool \textit{BPEL2oWFN}. Thus, any
\textit{BPEL} (\textit{Business Process Execution Language for Web Services}) \ref{} process can
easily be analyzed.

To compute the states of the graph nodes \textit{Fiona} uses the efficient algorithms that were
implemented in the model checking tool \textit{LoLA}.

\textit{Fiona} was written by Daniela Weinberg, Peter Massuthe, Karsten Wolf, Kathrin Kaschner,
Christian Gierds and Jan Bretschneider. It is part of the Tools4\textit{BPEL} project funded by the
German Bundesministerium f"ur Bildung und Forschung. See
@url{http://www.informatik.hu-berlin.de/top/tools4bpel} for details.

\section{Invoking Fiona}

A given \textit{oWFN} can be analyzed in two ways by \textit{Fiona}. Therefore the standard
invocation of \textit{Fiona} is:

\begin{itemize}
\item checking controllability: \texttt{fiona -n inputNet.owfn -t IG}
\item calculating the operating guideline: \texttt{fiona -n inputNet.owfn -t OG}
\end{itemize}

where \texttt{inputNet.owfn} contains an \textit{oWFN} written in the appropriate format
\texttt{File Format owfn}. The option \texttt{-t IG} lets \textit{Fiona} generate the interaction
graph of the given net. In case the graph's size is not too big, a png graphics is created that
shows the interaction graph. Further, an output is written on the command line indicating the size
of the graph and the statement whether the oWFN is controllable or not.

The option @samp{-t OG} lets \textit{Fiona} generate the operating guideline of the given
\textit{oWFN}. In case the graph's size is not too big, a png graphics is created that shows the
interaction graph. Further, an output is written on the command line indicating the size of the
graph and the statement whether the oWFN is controllable or not.

For more examples, see @ref{Examples}.

\textit{Fiona} can be called without any parameter. In this case, it calculates the interaction
graph of the \textit{oWFN}, that is being read from the standard input (\texttt{stdin}).

\subsection{Options}

\textit{Fiona} supports the following command-line options:

\begin{tabular}{|l|l|l|}
  \hline
 -h $|$ --help & print this information and exit\\
 -v $|$ --version & print version information and exit\\
 -d $|$ --debug=$<$level$>$& set debug $<$level$>$:\\
                     &              1 - show nodes and dfs information\\
                     &              2 - show analyse information (i.e. colors)\\
                     &              3 - show information on events and states\\
                     &              4 - yet to be defined ;)\\
                     &              5 - show detailed information on everything\\
\hline
\end{tabular}


\begin{tabular}{|l|ll|}
  \hline
 -t $|$ --type=$<$type$>$& \multicolumn{2}{l|}{select the modus operandi of fiona $<$type$>$:}\\
                      &           \multicolumn{2}{l|}{(only one type is allowed; default is IG)}\\
                      &             IG          &compute interaction graph\\
                      &             OG          &compute operating guideline\\
                      &             match       &check whether a given oWFN matches with a given OG\\
                      &             PV          &calculate the public view of a given OG (the result is an\\
                      &                         &automaton in OG file format)\\
                      &             productog   &calculate the product OG of all given OGs\\
                      &             simulation  &check whether the first OG characterizes more strategies\\
                      &                         &than the second one (if an oWFN is given, its OG\\
                      &                         &is computed automatically)\\
                      &             equivalence &check whether two OGs characterize the same\\
                      &                         &strategies (if an oWFN is given, its OG is computed\\
                      &                         &automatically) (option -b1 can be used to\\
                      &                         &check the equivalence of (already present!) BDDs)\\
                      &             filter      &reduces the first OG such that it simulates the second OG\\
                      &                         &(if possible)\\
                      &             isacyclic   &check a given OG for cycles\\
                      &             count       &count the number of strategies that are characterized by a\\
                      &                         &given OG\\
                      &             png         &generate png files for all given oWFNs\\
                      &             readOG      &only reads a given OG File\\
\hline
\end{tabular}


\begin{tabular}{|l|l|l|}
\hline
 -m $|$ --messagebound=$<$level$>$& set maximum number of same messages per\\
                      &           state to $<$level$>$  (default is 1)\\
 -r $|$ --reduceIG& use reduction rules for IG\\
 -R $|$ --reduce-nodes& use node reduction (IG or OG) which stores\\
                      &           less states per IG/OG node\\
                      &           (reduces memory, but increases time)\\
\hline
\end{tabular}

\begin{tabular}{|l|ll|}
  \hline
 -s $|$ --show=$<$parameter$>$&\multicolumn{2}{l|}{different display options $<$parameter$>$:}\\
                      &            blue       &show blue nodes only (default)\\
                      &                       &(empty node not shown)\\
                      &             empty     &show empty node\\
                      &             rednodes  &show blue and red nodes\\
                      &                       &(only empty node not shown\\
                      &             allnodes  &show all nodes\\
                      &             allstates &show all calculated states per\\
                      &                       &node\\
                      &             deadlocks &show all but transient states\\
\hline
\end{tabular}


\begin{tabular}{|l|l|l|}
  \hline
 -b $|$ --BDD=$<$reordering$>$& enable BDD construction (only relevant for OG)\\
                      &              $<$reordering$>$ specifies reodering method:\\
                      &              0 - CUDD\_REORDER\_SAME\\
                      &              1 - CUDD\_REORDER\_NONE\\
                      &              2 - CUDD\_REORDER\_RANDOM\\
                      &              3 - CUDD\_REORDER\_RANDOM\_PIVOT\\
                      &              4 - CUDD\_REORDER\_SIFT\\
                      &              5 - CUDD\_REORDER\_SIFT\_CONVERGE\\
                      &              6 - CUDD\_REORDER\_SYMM\_SIFT\\
                      &              7 - CUDD\_REORDER\_SYMM\_SIFT\_CONV\\
                      &              8 - CUDD\_REORDER\_WINDOW2\\
                      &              9 - CUDD\_REORDER\_WINDOW3\\
                      &             10 - CUDD\_REORDER\_WINDOW4\\
                      &             11 - CUDD\_REORDER\_WINDOW2\_CONV\\
                      &             12 - CUDD\_REORDER\_WINDOW3\_CONV\\
                      &             13 - CUDD\_REORDER\_WINDOW4\_CONV\\
                      &             14 - CUDD\_REORDER\_GROUP\_SIFT\\
                      &             15 - CUDD\_REORDER\_GROUP\_SIFT\_CONV\\
                      &             16 - CUDD\_REORDER\_ANNEALING\\
                      &             17 - CUDD\_REORDER\_GENETIC\\
                      &             18 - CUDD\_REORDER\_LINEAR\\
                      &             19 - CUDD\_REORDER\_LINEAR\_CONVERGE\\
                      &             20 - CUDD\_REORDER\_LAZY\_SIFT\\
                      &             21 - CUDD\_REORDER\_EXACT\\
 -B $|$ --OnTheFly=$<$reordering$>$&  BDD construction on the fly (only for OG)\\
                      &           $<$reordering$>$ as above\\
 -o $|$ --output=$<$filename$>$& prefix of the output files\\
 -Q $|$ --no-output& runs quietly, i.e., produces no output files\\
 -M $|$ --multipledeadlocks& create multiple deadlocks of public view\\
                      &          (only relevant for PV mode -t PV)\\
 -p $|$ --parameter=$<$param$>$& additional parameter $<$param$>$\\
                      &              no-png    - does not create a PNG file\\
                      &              diagnosis - disables optimizations\\
\hline
\end{tabular}


%Literaturverzeichnis
%\bibliographystyle{elsart-num}
%\bibliography{bibs}
%\bibliography{../bibs_mit-TB}


\end{document}
